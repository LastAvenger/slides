\documentclass[CJK,xetex]{beamer}
\usepackage{xltxtra,fontspec,xunicode}
\usepackage[slantfont,boldfont]{xeCJK}
%\usepackage[usenames,dvipsnames]{color}
%\usepackage{verbatim}
%\setCJKmainfont{AR PL UMing CN}
\setCJKmainfont{WenQuanYi Zen Hei}
%\setCJKmainfont{WenQuanYi Micro Hei}
%\setCJKmainfont{AR PL KaitiM GB}
\punctstyle{kaiming}
%\setmonofont{DejaVu Sans Mono}
%\setmainfont{WenQuanYi Zen Hei}

%\usepackage{tikz}
%\usetikzlibrary{arrows,backgrounds,shapes.arrows,decorations.pathmorphing,fit,chains,positioning}

\mode<presentation>
{
   \usetheme{Warsaw}
  %\usetheme{PaloAlto}
  % or ...

  \setbeamercovered{transparent}
  % or whatever (possibly just delete it)
}

%\usepackage{beamerthemesplit}
%\includeonlyframes{this}
%\newcommand{\tikzemph}[1]{\normalsize\textcolor{green!50!black}{#1}}
%\newcommand{\mynode}[2]{node[inner sep=1pt,anchor=north,rectangle split,rectangle split parts=2] {#1 \nodepart{second} #2}}
%\newcommand{\myannode}[2]{node[text width=30pt,inner sep=1pt,anchor=north,rectangle split,rectangle split parts=2] {#1 \nodepart{second} #2}}
%\newcommand{\mynwnode}[2]{node[inner sep=1pt,anchor=north west,rectangle split,rectangle split parts=2] {#1 \nodepart{second} #2}}
%\newcommand{\mynenode}[2]{node[inner sep=1pt,anchor=north east,rectangle
%  split,rectangle split parts=2] {#1 \nodepart{second} #2}}
\newcommand{\mydir}[1]{{\color{red} \texttt{#1}}}
\newcommand{\mycmd}[1]{{\color{blue} \texttt{#1}}}
\newcommand{\myfile}[1]{{\color{purple} \texttt{#1}}}
\title{Linux目录架构与包管理系统}
\author[李瑞彬]{李瑞彬(cheeselee@fedoraproject.org)}
\date{2011年8月5日}


%\AtBeginSubsection[]
%{
  %\begin{frame}<beamer>{Outline}
    %\tableofcontents[sectionstyle=show/hide,subsectionstyle=show/shaded/hide] %currentsection,currentsubsection,hideallsubsections]
  %\end{frame}
%}
%\AtBeginSection[]
%{
  %\begin{frame}<beamer>{Outline}
    %\tableofcontents[currentsection,hideothersubsections]
  %\end{frame}
%}
%\newtheorem{mydef}{定义}
%\newtheorem{definition}{

\begin{document}

%\frame[plain]{\titlepage}
\frame{\titlepage}

%\begin{frame}[plain]{Outline}
%  \tableofcontents[hideallsubsections]
  % You might wish to add the option [pausesections]
%\end{frame}
\section{Unix目录结构}
\begin{frame}{一棵树}
\begin{itemize}[<+->]
  \item 根目录 ``\mydir{/}''
  \item DOS每个磁盘分区一棵树
  \item 在Unix中,磁盘分区要使用,要先挂载(\texttt{mount})到目录树上
  \item 用户不需要理会机器的分区策略
  \item \texttt{df} 命令
\end{itemize}
\end{frame}

\begin{frame}{关键思想:依功能摆放文件}
  \begin{itemize}[<+->]
  \item \mydir{bin/}:可执行文件
  \item \mydir{lib/}:依赖机器架构的库文件
  \item \mydir{/usr/share/}:不依赖机器架构的文件
  \item \mydir{/etc/}:系统级配置文件
  \item \mydir{/var/}:常变动的文件,如程序数据、日志  
  \end{itemize}
\end{frame}

\begin{frame}{一些重要的目录}
\begin{itemize}[<+->]
  \item \mydir{/bin/}:Unix基础程序
  \item \mydir{/sbin/}:需root权限运行的可执行文件
  \item \mydir{/lib/}:基础C库
  \item \mydir{/etc/} \mydir{/var/}
  \item \mydir{/dev/}:以文件形式呈现出来的硬件设备
  \item \mydir{/home/*/} \mydir{/root/} 用户目录
  \item \mydir{/mnt/} \mydir{/media/} 挂载临时分区
  \item \mydir{/proc/} 系统运行状态信息
  \item \mydir{/boot/} 启动管理器配置与Linux内核
  \item \mydir{/usr/} 非基础级程序,内容与根目录有一些相似,安装发行版打包的程序
  \item \mydir{/usr/local/} 与\mydir{/usr}雷同,手动编译安装的程序默认目录
\end{itemize}
\end{frame}

\begin{frame}{所谓安装删除程序}
  \begin{itemize}[<+->]
  \item 安装程序就是将它的可执行文件复制到\mydir{/usr/bin/},将库文件复制到\mydir{/usr/lib/},将配置文件复制到\mydir{/etc/},……
  \item 那要删除程序怎么办呢?
    \item 依赖的地狱!
    \item 包管理系统就应运而生
    \item 包管理系统包括若干个组成程序
      \begin{itemize}
      \item 包管理器(package manager):如\myfile{rpm}、\myfile{dpkg}
      \item 依赖处理系统:如\myfile{yum}、\myfile{apt}
      \item 交互式操作界面:如\myfile{PackageKit}、\myfile{synaptic}、\myfile{aptitude}
      \end{itemize}
  \end{itemize}
\end{frame}

\section{包管理器(package manager)}

\begin{frame}{包管理器(package manager)}
\begin{itemize}[<+->]
  \item 包管理器基本功能:
  \begin{itemize}
    \item 从软件包提取文件复制到系统目录(安装程序)
    \item 记录文件属于哪个软件包
    \item 删除软件包所属的所有文件(删除程序)
    \item 管理配置文件
    \item 在程序安装前后运行必要的脚本
    \end{itemize}
    \item 包管理系统是Linux发行版之间的最主要差别
  \item RedHat系(RHEL/CentOS Fedora):\myfile{rpm}
    \item Debian系(Debian Ubuntu):\myfile{dpkg}
\end{itemize}
\end{frame}

\begin{frame}{软件包}
\begin{itemize}[<+->]
  \item 将要安装的文件以绝对路径打包压缩到一起
  \item 记录软件信息(包名、版本、依赖关系……)
  \item 实例:\myfile{nginx-0.8.54-1.fc15.x86$\_$64.rpm} \myfile{nginx$\_$0.7.67-3$\_$i386.deb}
\end{itemize}
\end{frame}

\section{依赖处理程序}
\begin{frame}{依赖处理程序}
\begin{itemize}[<+->]
  \item 基本功能:
\begin{itemize}
  \item 安装软件(从软件库拉取软件包及它的依赖的软件包,然后让包管理器有序安装它们)
  \item 删除软件(并自动删除一些无用的依赖包)
  \item 更新软件(将已安装的软件升级到软件库里的最新版本)
\end{itemize}
  \item RedHat系(RHEL/CentOS Fedora):\myfile{yum}
    \item Debian系(Debian Ubuntu):\myfile{apt}
\end{itemize}
\end{frame}

\section{Yum}
\begin{frame}{Yum}
  \begin{itemize}
  \item 安装软件(根据包名):\mycmd{yum install nginx}
  \item 安装软件(根据程序文件名):\\ \mycmd{yum install /usr/bin/eclipse}
  \item 删除软件:\mycmd{yum remove httpd}
  \item 删除软件及无用的依赖(需要remove-with-leaves插件):\\ \mycmd{yum remove --remove-leaves httpd}
  \item 历史回退:\mycmd{yum history undo last}
  \item 参考\mycmd{yum --help}和\mycmd{man yum}
  \end{itemize}
\end{frame}

\section{Aptitude}
\begin{frame}{Aptitude}
  \begin{itemize}
    \item 前言:用\myfile{apt-spy}自动选择最快的源
    \item aptitude是apt的一个神级的交互操作界面
  \end{itemize}
\end{frame}
\end{document}
