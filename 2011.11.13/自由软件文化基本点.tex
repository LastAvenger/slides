\documentclass[handout,CJK,xetex]{beamer}
\usepackage{xltxtra,fontspec,xunicode}
\usepackage[slantfont,boldfont]{xeCJK}
%\usepackage[usenames,dvipsnames]{color}
%\usepackage{verbatim}
%\setCJKmainfont{AR PL UMing CN}
\setCJKmainfont{WenQuanYi Zen Hei}
%\setCJKmainfont{WenQuanYi Micro Hei}
%\setCJKmainfont{AR PL KaitiM GB}
\punctstyle{quanjiao}
%\setmonofont{DejaVu Sans Mono}
%\setmainfont{WenQuanYi Zen Hei}
\usepackage{hyperref}
%\usepackage{tikz}
%\usetikzlibrary{arrows,backgrounds,shapes.arrows,decorations.pathmorphing,fit,chains,positioning}

\mode<presentation>
{
   \usetheme{Warsaw}
  %\usetheme{PaloAlto}
  % or ...

  \setbeamercovered{transparent}
  % or whatever (possibly just delete it)
}

%\usepackage{beamerthemesplit}
%\includeonlyframes{this}
%\newcommand{\tikzemph}[1]{\normalsize\textcolor{green!50!black}{#1}}
%\newcommand{\mynode}[2]{node[inner sep=1pt,anchor=north,rectangle split,rectangle split parts=2] {#1 \nodepart{second} #2}}
%\newcommand{\myannode}[2]{node[text width=30pt,inner sep=1pt,anchor=north,rectangle split,rectangle split parts=2] {#1 \nodepart{second} #2}}
%\newcommand{\mynwnode}[2]{node[inner sep=1pt,anchor=north west,rectangle split,rectangle split parts=2] {#1 \nodepart{second} #2}}
%\newcommand{\mynenode}[2]{node[inner sep=1pt,anchor=north east,rectangle
%  split,rectangle split parts=2] {#1 \nodepart{second} #2}}
\newcommand{\mydir}[1]{{\color{red} \texttt{#1}}}
\newcommand{\mycmd}[1]{{\color{blue} \texttt{#1}}}
\newcommand{\myfile}[1]{{\color{purple} \texttt{#1}}}
\title{自由软件文化基本点}
\author[李瑞彬]{李瑞彬(cheeselee@fedoraproject.org)}
\date{2011年11月13日}


%\AtBeginSubsection[]
%{
  %\begin{frame}<beamer>{Outline}
    %\tableofcontents[sectionstyle=show/hide,subsectionstyle=show/shaded/hide] %currentsection,currentsubsection,hideallsubsections]
  %\end{frame}
%}
\AtBeginSection[]
{
  \begin{frame}<beamer>{四个基本点}
    \tableofcontents[currentsection,hideothersubsections]
  \end{frame}
}
%\newtheorem{mydef}{定义}
%\newtheorem{definition}{

\begin{document}

%\frame[plain]{\titlepage}
\frame{\titlepage}

\begin{frame}{Fedora Packaging Guidelines}
  \begin{itemize}
  \item \url{fedoraproject.org/wiki/Packaging/Guidelines}
  \item 规定怎样的软件可以进入Fedora主库
  \item 后面简称为Guideline
  \end{itemize}
\end{frame}


\begin{frame}{四个基本点}
  \tableofcontents[hideallsubsections]
  % You might wish to add the option [pausesections]
\end{frame}
%\section*{}


\section{遵从标准}
\begin{frame}{必须的标准}
\begin{itemize}[<+->]
  \item 比如:Filesystem Hierarchy Standard (FHS)
  \item \url{http://www.pathname.com/fhs/}
    \begin{block}{Guideline}
      Fedora packages must follow the FHS.
    \end{block}
  \item 这不是重点。
\end{itemize}
\end{frame}

\begin{frame}{建议的标准}
\begin{itemize}[<+->]
  \item 比如:MPRIS
  \item \url{http://www.freedesktop.org/wiki/Specifications/mpris-interfacing-specification}
  \item 媒体播放器的通用D-Bus接口
  \item Amarok, Exaile, VLC, $\cdots$
  \item 自由软件社区更容易达成和支持这类标准
\end{itemize}
\end{frame}

\section{便于编译时定制与测试}
\begin{frame}{自动构建系统}
  \begin{itemize}[<+->]
  \item 自由软件应该方便让使用者设置编译选项,特别是编译器的优化级
  \item 自由软件应该采用某种自由的自动构建系统(build automation software)。
  \item \url{en.wikipedia.org/wiki/List_of_build_automation_software}
  \item Autoconf/Automake, CMake, Maven, $\cdots$
  \item 反面教材:~\href{http://www.xmind.net/}{XMind}
  \item 自由软件应该提供配套测试,并且可以简单运行(make test)
    \begin{block}{Guideline}
      If the source code of the package provides a test suite, it should be executed $\cdots$
    \end{block}
  \end{itemize}
\end{frame}

\section{不要重复发明轮子}
\begin{frame}{Don't reinvent the wheel!}
  \begin{itemize}[<+->]
  \item 自由软件应尽量依赖现成的自由软件
  \item 依赖越多,开发者视野越广,软件质量越高
  \item 依赖越多,打包者越兴奋
  \item 开发者应及时更新依赖
  \item 让你的发明被依赖!
  \item 自由软件应提供足够的文档,让别人也同样容易地整合你的成果
    \begin{block}{Guideline}
Any relevant documentation included in the source distribution should be included in the package $\ldots$
    \end{block}
  \end{itemize}
\end{frame}

\section{便于参与}
\begin{frame}{版本控制(version control)和缺陷跟踪(bug tracking)}
  \begin{itemize}[<+->]
  \item 两者为最基本的社区沟通工具
  \item 使用Google Code, Github, $\cdots$
  \item 版本控制系统展现出项目开发状态,便于生成补丁
  \item 缺陷跟踪系统便于用户和下游反馈问题,便于制定开发计划
    \begin{block}{Guideline}
All patches should have an upstream bug link or comment
\end{block}
\end{itemize}
\end{frame}

%\section*{}

\end{document}
